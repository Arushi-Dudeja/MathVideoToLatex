\documentclass[12pt]{article}
\usepackage[utf8]{inputenc}
\usepackage[spanish]{babel}
\usepackage{amsmath}
\usepackage{amssymb}
\usepackage{graphicx}
\begin{document}

\title{Sistemas de ecuaciones en forma matricial}
\author{Profesor Postales}
\date{\today}

\maketitle

\section{Introducción}

En esta lección, vamos a explicar cómo representar un sistema de ecuaciones en forma matricial y cómo resolverlo usando esa representación.

\section{Representación matricial de un sistema de ecuaciones}

Un sistema de ecuaciones lineales con $n$ ecuaciones y $m$ incógnitas se puede representar en forma matricial como:

$$\begin{bmatrix} a_{11} & a_{12} & \cdots & a_{1m} \\\ a_{21} & a_{22} & \cdots & a_{2m} \\\ \vdots & \vdots & \ddots & \vdots \\\ a_{n1} & a_{n2} & \cdots & a_{nm} \end{bmatrix} \begin{bmatrix} x_1 \\\ x_2 \\\ \vdots \\\ x_m \end{bmatrix} = \begin{bmatrix} b_1 \\\ b_2 \\\ \vdots \\\ b_n \end{bmatrix}$$

donde:

* $A$ es la matriz de coeficientes, una matriz $n \times m$ que contiene los coeficientes de las incógnitas.
* $X$ es la matriz de incógnitas, una matriz $m \times 1$ que contiene las incógnitas del sistema.
* $B$ es la matriz de términos independientes, una matriz $n \times 1$ que contiene los términos independientes del sistema.

\section{Resolución de sistemas de ecuaciones en forma matricial}

Para resolver un sistema de ecuaciones en forma matricial, podemos utilizar la siguiente fórmula:

$$X = A^{-1}B$$

donde $A^{-1}$ es la matriz inversa de $A$.

\subsection{Cálculo de la matriz inversa}

La matriz inversa de una matriz cuadrada $A$ es una matriz $A^{-1}$ tal que:

$$AA^{-1} = A^{-1}A = I$$

donde $I$ es la matriz identidad.

La matriz inversa de una matriz $A$ se puede calcular usando la fórmula de Cramer:

$$A^{-1} = \frac{C^{adj}}{det(A)}$$

donde:

* $C^{adj}$ es la matriz adjunta de $A$, que es la transpuesta de la matriz de cofactores de $A$.
* $det(A)$ es el determinante de $A$.

\subsection{Ejemplo}

Consideremos el siguiente sistema de ecuaciones:

$$\begin{align*}
2x + 5y &= 20 \\\
-x + 4y &= 7
\end{align*}$$

Podemos representar este sistema en forma matricial como:

$$\begin{bmatrix} 2 & 5 \\\ -1 & 4 \end{bmatrix} \begin{bmatrix} x \\\ y \end{bmatrix} = \begin{bmatrix} 20 \\\ 7 \end{bmatrix}$$

Calculando la matriz inversa de la matriz de coeficientes, obtenemos:

$$A^{-1} = \frac{1}{13}\begin{bmatrix} 4 & -5 \\\ 1 & 2 \end{bmatrix}$$

Sustituyendo en la fórmula de resolución, obtenemos:

$$\begin{bmatrix} x \\\ y \end{bmatrix} = \frac{1}{13}\begin{bmatrix} 4 & -5 \\\ 1 & 2 \end{bmatrix} \begin{bmatrix} 20 \\\ 7 \end{bmatrix} = \begin{bmatrix} \frac{45}{13} \\\ \frac{34}{13} \end{bmatrix}$$

Por lo tanto, la solución del sistema de ecuaciones es $x = \frac{45}{13}$ e $y = \frac{34}{13}$.

\section{Conclusión}

La representación matricial de los sistemas de ecuaciones permite resolverlos utilizando métodos algebraicos, como el cálculo de la matriz inversa. Este método es especialmente útil para resolver sistemas de ecuaciones grandes o complejos.

\end{document}