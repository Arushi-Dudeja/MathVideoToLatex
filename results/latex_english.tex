\documentclass[12pt]{article}
\usepackage{amsmath}
\usepackage{graphicx}
\begin{document}
\section{Systems of Equations in Matrix Form}
\subsection{Introduction}
A system of equations is a set of two or more equations that have the same variables. Systems of equations can be solved using a variety of methods, including substitution, elimination, and matrix inversion.
\subsection{Matrix Form of a System of Equations}
A system of $m$ equations in $n$ variables can be written in matrix form as follows:
\begin{align}
\begin{bmatrix}
a_{11} & a_{12} & \cdots & a_{1n} \\
a_{21} & a_{22} & \cdots & a_{2n} \\
\vdots & \vdots & \ddots & \vdots \\
a_{m1} & a_{m2} & \cdots & a_{mn}
\end{bmatrix}
\begin{bmatrix}
x_1 \\
x_2 \\
\vdots \\
x_n
\end{bmatrix}
=
\begin{bmatrix}
b_1 \\
b_2 \\
\vdots \\
b_m
\end{bmatrix}
\end{align}
where $A$ is the $m \times n$ coefficient matrix, $x$ is the $n \times 1$ column vector of variables, and $b$ is the $m \times 1$ column vector of constants.
\subsection{Solving Systems of Equations in Matrix Form}
There are a number of methods for solving systems of equations in matrix form. One common method is to use matrix inversion. To solve a system of equations using matrix inversion, we first find the inverse of the coefficient matrix $A^{-1}$. We then multiply both sides of the equation by $A^{-1}$ to get:
\begin{align}
A^{-1}Ax &= A^{-1}b \\
x &= A^{-1}b
\end{align}
We can then use the inverse of the coefficient matrix to find the solution to the system of equations.
\subsection{Example}
Let's solve the following system of equations using matrix inversion:
\begin{align}
2x + 5y &= 20 \\
-x + 4y &= 7
\end{align}
First, we write the system of equations in matrix form:
\begin{align}
\begin{bmatrix}
2 & 5 \\
-1 & 4
\end{bmatrix}
\begin{bmatrix}
x \\
y
\end{bmatrix}
=
\begin{bmatrix}
20 \\
7
\end{bmatrix}
\end{align}
Next, we find the inverse of the coefficient matrix:
\begin{align}
A^{-1} = \frac{1}{\det(A)}
\begin{bmatrix}
d & -b \\
-c & a
\end{bmatrix}
\end{align}
where $\det(A)$ is the determinant of the coefficient matrix, and $a$, $b$, $c$, and $d$ are the elements of the coefficient matrix.
In this case, the determinant of the coefficient matrix is:
\begin{align}
\det(A) = (2)(4) - (-1)(5) = 13
\end{align}
And the inverse of the coefficient matrix is:
\begin{align}
A^{-1} = \frac{1}{13}
\begin{bmatrix}
4 & -5 \\
1 & 2
\end{bmatrix}
\end{align}
Finally, we multiply both sides of the equation by $A^{-1}$ to get:
\begin{align}
A^{-1}Ax &= A^{-1}b \\
x &= A^{-1}b \\
&= \frac{1}{13}
\begin{bmatrix}
4 & -5 \\
1 & 2
\end{bmatrix}
\begin{bmatrix}
20 \\
7
\end{bmatrix} \\
&= \frac{1}{13}
\begin{bmatrix}
(4)(20) + (-5)(7) \\
(1)(20) + (2)(7)
\end{bmatrix} \\
&= \frac{1}{13}
\begin{bmatrix}
55 \\
34
\end{bmatrix} \\
&=
\begin{bmatrix}
\frac{55}{13} \\
\frac{34}{13}
\end{bmatrix}
\end{align}
Therefore, the solution to the system of equations is $x = \frac{55}{13}$ and $y = \frac{34}{13}$.
\end{document}